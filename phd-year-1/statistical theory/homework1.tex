\documentclass[12pt]{article}
\usepackage{amsmath,amssymb}
\usepackage[a4paper,bindingoffset=0.2in,%
left=0.8in,right=0.8in,top=1in,bottom=1in,%
footskip=.25in]{geometry}

\title{Statistical Theory Homework 1}
\date{\today}
\author{Bohao Tang}

\begin{document}

\maketitle

\begin{enumerate}
    \item
    \begin{enumerate}
        \item $\left\{ (0,g),(0,f),(0,s),(1,g),(1,f),(1,s) \right\}$
        \item $\left\{ (0,s),(1,s) \right\}$
        \item $\left\{ (0,g),(0,f),(0,s) \right\}$
        \item $\left\{ (0,s),(1,s),(1,g),(1,f) \right\}$
        \item $\left\{ (1,s) \right\}$
    \end{enumerate}
    \item 
    \begin{enumerate}
        \item $A \cap B^c \cap C^c$
        \item $A \cap B \cap C^c$
        \item $A \cap B \cap C$
        \item $A \cup B \cup C$
        \item $(A \cap B \cap C^c) \cup (A \cap B^c \cap C) \cup (A^c \cap B \cap C)$
        \item $A^c \cap B^c \cap C^c$        
    \end{enumerate}
    \item 
    \begin{enumerate}
        \item $\textbf{P}$( at least a 6 ) $= \frac{6 + 6 - 1}{6 \times 6} = \frac{11}{36}$
        \item $\textbf{P}$( same number ) $= \frac{6}{6 \times 6} = \frac{1}{6}$ 
    \end{enumerate}
    \item 
    It's the same number as to choose $2$ people out of $20$ to shake hands, and that's $\binom{20}{2} = 190$.
    \item
    \begin{enumerate}
        \item $\frac{3 \times 4}{\binom{5}{2} \times 2 !} = \frac{3}{5}$
        \item $\frac{2 \times 3 + 3 \times 2}{\binom{5}{2} \times 2 !} = \frac{3}{5}$
        \item $\frac{\binom{3}{2} \times 2 !}{\binom{5}{2} \times 2 !} = \frac{3}{10}$
    \end{enumerate}
    \item I think it's to choose $n$ ordered character from a-z , where $n$ is the sum of the numbers of one's last and given name. Then:
    \begin{enumerate}
        \item $26 \times 26 \times 26 = 17576$
        \item $26 \times 26 + 26 \times 26 \times 26 = 18252$
        \item $26 \times 26 + 26 \times 26 \times 26 + 26 \times 26 \times 26 \times 26 = 475228$
    \end{enumerate}
    \item 
    We can deal $\{1,2,3\}$ as one number and arrange the $n-2$ numbers in order first, then we arrange ${1,2,3}$ in order. Therefore:
    $\textbf{P} = \frac{ ( n - 2 ) ! \times 3 ! }{ n ! } = \frac{6}{n ( n - 1 )}$ , where $n$ must no less than $3$. 
    \item 
    It's equivalent to choose $7$ children from $10$ to have one and only one gift, therefore the number of results is $\binom{10}{3} = 120$.
    \item 
    It's the sum of ways to choose $t$ elements in $(x_1,x_2,\cdots,x_n)$ to be $1$ , where $t$ ranges from $k$ to $n$. And that's $\sum_{t=k}^{n} \binom{n}{t}$
    \item 
    $\textbf{P} = \frac{|\text{choose $r$ spaces occupied out of the rest $N-2$ spaces}|}{|\text{choose $r$ occupied out of N}|} = \frac{\binom{N - 2}{r}}{\binom{N}{r}} = \frac{(N - r) (N - r - 1)}{N (N - 1)}$
    \item I'm a little confused about the meaning of "double" here, I think it means 2 dices are in same number.
    \begin{enumerate}
        \item $\textbf{P} = \frac{2}{6 \times 6} / \frac{6}{6 \times 6} = \frac{1}{3}$
        \item $\textbf{P} = \frac{5 + 5}{6 \times 6} / \frac{6 \times 6 - 6}{6 \times 6} = \frac{1}{3}$
    \end{enumerate}
    \item 
    $\textbf{P}(\text{coin with heads and tails} | \text{show up heads}) = $\\
    $\textbf{P}(\text{choosed third coin and tossed to be heads}) / \textbf{P}(\text{tossed to be heads}) = $\\
    $\frac{1/3 \times 1/2}{1/3 \times 1/2 + 1/3 \times 1 + 1/3 \times 0} = \frac{1}{3}$
    \item 
    if the $n^{th}$ stamp is of type $i$, then it's new means the first $n - 1$ stamps are of types other than $i$, whose probability is $(1 - p_i)^{n-1}$. Therefore, 
    $$\textbf{P}(\text{new}) = \sum_1^m \textbf{P}(\text{new} | \text{$n^{th}$ of type i}) \textbf{P}(\text{$n^{th}$ of type i}) = \sum_1^m p_i (1-p_i)^{n-1}$$
\end{enumerate}

\section*{\normalsize{Extra Credit:}}

Here and in the question, I think, "has k balls" means having exactly $k$ balls.

Denote $A_{n,r,k}$ to be the number of different results of putting $n$ balls in $r$ urns such that exactly one urn has $k$ balls; $B_{n,r,k}$ to be that of putting $n$ balls in $r$ urns such that no urn has $k$ balls;  $C_{n,r,k}$ to be that of putting $n$ balls in $r$ urns such that at least one urn has $k$ balls; and $N_{n,r}$ to be the number of methods to put $n$ balls in $r$ urns.

Then, first it's easy to see:

\begin{eqnarray} \label{Binit}
B_{n,r,k} &=& N_{n,r} \hspace{1.5em} \text{if $n < k$} \\
B_{n,r,k} &=& 1 \hspace{3em} \text{if $n \neq k$ and $r = 1$} \\
B_{n,r,k} &=& 0 \hspace{3em} \text{if $n = k$ and $r = 1$} \\
B_{n,r,k} &=& N_{n,r} - r \hspace{1.5em} \text{if $n = k$} 
\end{eqnarray}

\begin{equation} \label{Ninit}
N_{n,r} = \binom{n+r-1}{n}
\end{equation}

and 
\begin{equation} \label{BC}
B_{n,r,k} + C_{n,r,k} = N_{n,r}
\end{equation}

Also consider situations in $A_{n,r,k}$, it's obviously equivalent to choose one urn out of $r$ and put $k$ balls there, and then put the rest $n-k$ balls in rest $r-1$ urns such that no one has $k$ balls. Therefore we have:
\begin{equation} \label{A} 
A_{n,r,k} = \binom{r}{1} B_{n-k,r-1,k}
\end{equation}

and then we give an iterative expression of $C_{n,r,k}$. We can discuss which urn firstly has $k$ balls. If the $1^{st}$ urn firstly have $k$ balls then it's equivalent to put rest $n-k$ balls in rest $r-1$ urns without restriction. If it's the last urn to firstly have $k$ balls, then it's equivalent to put rest $n-k$ balls in first $r-1$ urns such that no one has $k$ balls.

Also, if $b^{th}$ urn firstly have $k$ balls (where $2 \le b \le r-1$), then we discuss how many balls are in first $b-1$ urns. Therefore it's the sum of "put $t$ balls in $b-1$ urns such that no one has $k$ balls and put $n-k-t$ balls in rest $r-b$ urns without restriction " where $t$ range from $0$ to $n-k$. Therefore:
\begin{equation} \label{C} 
C_{n,r,k} = N_{n-k,r-1} + \sum^{r-1}_{b=2} \sum_{t=0}^{n-k} B_{t,b-1,k} N_{n-k-t,r-b} + B_{n-k,r-1,k}
\end{equation}

Initial values in equations \ref{Binit} to \ref{Ninit} and iterative equations \ref{BC} to \ref{C} form a sufficient way to calculate $A_{n,r,k}$ and finally the probability will be:

$$\textbf{P} = A_{n,r,k} / N_{n,r}$$

\end{document}