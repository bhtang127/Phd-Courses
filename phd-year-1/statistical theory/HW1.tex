\documentclass[11pt]{article}
\usepackage{amsmath,amssymb}

\begin{document}
Here and in the question, I think, "has k balls" means having exactly $k$ balls.

Denote $A_{n,r,k}$ to be the number of different results of putting $n$ balls in $r$ urns such that exactly one urn has $k$ balls; $B_{n,r,k}$ to be that of putting $n$ balls in $r$ urns such that no urn has $k$ balls;  $C_{n,r,k}$ to be that of putting $n$ balls in $r$ urns such that at least one urn has $k$ balls; and $N_{n,r}$ to be the number of methods to put $n$ balls in $r$ urns.

Then, first it's easy to see:

\begin{eqnarray} \label{Binit}
B_{n,r,k} &=& N_{n,r} \hspace{1.5em} \text{if $n < k$} \\
B_{n,r,k} &=& 1 \hspace{3em} \text{if $n \neq k$ and $r = 1$} \\
B_{n,r,k} &=& 0 \hspace{3em} \text{if $n = k$ and $r = 1$} \\
B_{n,r,k} &=& N_{n,r} - r \hspace{1.5em} \text{if $n = k$} 
\end{eqnarray}

\begin{equation} \label{Ninit}
N_{n,r} = \binom{n+r-1}{n}
\end{equation}

and 
\begin{equation} \label{BC}
B_{n,r,k} + C_{n,r,k} = N_{n,r}
\end{equation}

Also consider situations in $A_{n,r,k}$, it's obviously equivalent to choose one urn out of $r$ and put $k$ balls there, and then put the rest $n-k$ balls in rest $r-1$ urns such that no one has $k$ balls. Therefore we have:
\begin{equation} \label{A} 
A_{n,r,k} = \binom{r}{1} B_{n-k,r-1,k}
\end{equation}

and then we give an iterative expression of $C_{n,r,k}$. We can discuss which urn firstly has $k$ balls. If the $1^{st}$ firstly have $k$ balls then it's equivalent to put rest $n-k$ balls in rest $r-1$ urns without restriction. If it's the last urn to firstly have $k$ balls, then it's equivalent to put rest $n-k$ balls in first $r-1$ urns such that no one has $k$ balls.

Also, if $b^{th}$ urn firstly have $k$ balls (where $2 \le b \le r-1$), then we discuss how many balls are in first $b-1$ urns. Therefore it's the sum of "put $t$ balls in $b-1$ urns such that no one has $k$ balls and put $n-k-t$ balls in rest $r-b$ urns without restriction " where $t$ range from $0$ to $n-k$. Therefore:
\begin{equation} \label{C} 
C_{n,r,k} = N_{n-k,r-1} + \sum^{r-1}_{b=2} \sum_{t=0}^{n-k} B_{t,b-1,k} N_{n-k-t,r-b} + B_{n-k,r-1,k}
\end{equation}

Initial values in equations \ref{Binit} to \ref{Ninit} and iterative equations \ref{BC} to \ref{C} form a sufficient way to calculate $A_{n,r,k}$ and finally the probability will be:

$$\textbf{P} = A_{n,r,k} / N_{n,r}$$

\end{document}